\chapter{Introduzione}\label{ch:introduzione}
Le intelligente artificiali sono riuscite al giorno d'oggi a risolvere innumerevoli problemi della vita reale come ad esempio il controllo avanzato di alcuni sistemi di automazione, le telecomunicazioni e la robotica. Anche in campo mobile vi sono molti servizi che incorporano IA che sfruttano i dati degli utenti al fine di proporre sistemi altamente personalizzati per le specifiche dell'utente come ad esempio la stessa tastiera qwerty dei nostri dispositivi che è in grado di memorizzare come scriviamo e 'impara' a proporre le prossime parole. Sfruttando le caratteristiche personali dei vari utente  non solo viene migliorata l'esperienza personale ma aiutano anche a controllare meglio i loro dispositivi. \\
La crescente preoccupazione per la privacy dei dati nei vari framework di machine learning esistenti ha alimentato un crescente interesse nello sviluppo di paradigmi di machine learning distribuiti come i framework di apprendimento federato che illustreremo in seguito.