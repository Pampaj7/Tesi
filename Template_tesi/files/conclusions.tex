\chapter{Conclusioni}\label{ch:conclusioni}
La nuova filosofia Dem-AI ha fornito linee guida generali per meccanismi di strutturazione gerarchica specializzati, generalizzati e auto-organizzanti in sistemi di machine learning distribuiti su larga scala. Ispirati da queste linee guida, abbiamo formulato i problemi gerarchici di apprendimento generalizzato e sviluppato un nuovo algoritmo di apprendimento distribuito, DemLearn. In questo lavoro, basato sulla somiglianza nelle caratteristiche di apprendimento, il clustering agglomerativo consente l'auto-organizzazione degli agenti di apprendimento in una struttura gerarchica, che viene aggiornata periodicamente. L'analisi dettagliata delle valutazioni sperimentali ha mostrato i vantaggi e gli svantaggi dell'algoritmo proposto. Rispetto al FL convenzionale, mostriamo che DemLearn in modo significativo migliora le prestazioni di generalizzazione dei modelli client senza compromettere in gran parte le prestazioni di specializzazione dei modelli client. Di conseguenza, DemLearn consente buone capacità di apprendimento del compromesso dei modelli client con prestazioni C-SPE e C-GEN elevate, mentre altri algoritmi possono produrre solo modelli locali distorti con capacità generalizzate basse. Queste osservazioni favoriscono una migliore comprensione e un miglioramento delle prestazioni di specializzazione e generalizzazione dei modelli di apprendimento nei futuri sistemi Dem-AI. L'apprendimento democratizzato fornisce ingredienti unici per sviluppare futuri sistemi intelligenti personalizzati distribuiti.
A tal fine, il progetto di apprendimento potrebbe essere ulteriormente studiato con set di dati personalizzati, esteso per capacità di apprendimento multi-task e convalidato con una generalizzazione effettiva per i nuovi utenti e cambiamenti ambientali. Trasformando in realtà i sistemi generali di apprendimento distribuito, il Dem-AI deve essere analizzato in profondità da una varietà di prospettive come la robustezza e la diversità dei modelli di apprendimento e i nuovi meccanismi di trasferimento e distillazione della conoscenza. Inoltre, è possibile incorporare il design flessibile con approcci attuali come meta-learning e metodi basati sull'ottimizzazione per migliorare ulteriormente la personalizzazione in FL.