\chapter{Algoritmo di apprendimento federato}\label{ch:chapter1}
Il Federated Learning è una tecnologia di apprendimento automatico distribuito che permette di addestrare un modello di intelligenza artificiale su un gran numero di dispositivi, mantenendo i dati su ciascun dispositivo.
FL (federated learning) è strutturato in modo che gli agenti di apprendimento si coordinino tramite un server centrale per addestrare un modello di apprendimento globale in modo distribuito. Questi agenti ricevono il modello di apprendimento locale in base ai set di dati disponibili. Quindi restituiscono i modelli di apprendimento aggiornati al server per l'aggiornamento del modello globale tramite un'operazione di aggregazione senza rivelare i dati di addestramento privato agli altri. In pratica, il set di dati privato raccolto presso ciascun agente è sbilanciato, altamente personalizzato per alcune applicazioni come la scrittura a mano e il riconoscimento vocale e presenta caratteristiche non indipendenti e non identicamente distribuite. Pertanto il processo iterativo di aggiornamento del modello globale migliora la generalizzazione del modello ma danneggia anche le prestazioni personalizzate degli agenti. Quindi gli algoritmi FL esistenti non possono gestire in modo efficiente la relazione coesiva tra la capacità di generalizzazione e personalizzazione del modello di apprendimento addestrato. Vi sono stati alcuni tentativi di studiare e migliorare le prestazioni personalizzate di FL utilizzando un algoritmo di media federata personalizzata (Per-FedAvg) basato su un framework di meta-apprendimento. Si è anche provato ad avere un framework personalizzato adattivo in cui è stata adottata una combinazione di modelli locali e globali per ridurre l'errore di generalizzazione. Tuttavia in entrambi i casi la relazione coesiva tra generalizzazione e personalizzazione non è stata adeguatamente analizzata.\\
In questo elaborato svilupperemo un nuovo framework di apprendimento distribuito che può estendere direttamente lo schema FL convenzionale per risolvere collettivamente un compito di apprendimento comune ai vari agenti. Diversamente dagli algoritmi FL esistenti per la costruzione di un singolo modello generalizzato, manteniamo modelli di gruppo gerarchici autoorganizzanti. Di conseguenza adottiamo il clustering gerarchico agglomerativo e aggiorniamo periodicamente la struttura gerarchica in base alla somiglianza nelle caratteristiche di apprendimento degli utenti. In particolare proponiamo la generalizzazione gerarchica in forma ricorsiva. Per risolvere il complesso problema dato dalla forma ricorsiva, consideriamo  l'algoritmo di apprendimento distribuito: DemLearn. 